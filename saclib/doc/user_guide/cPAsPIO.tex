In this section we will describe the polynomial input and output
routines that are available in \saclib.  Before proceeding further,
the reader should be familiar with the internal representations of
polynomials which are discussed in Section \ref{c:PA s:I ss:D}.

\subsection{Recursive polynomials over $\BbbZ$}
The {\em external canonical representation} of sparse {\em recursive}
polynomials over $\BbbZ$ is defined by the following rules.  First of
all, each polynomial is enclosed in parentheses.  A term is
represented by the coefficient immediately followed by the variable
(no space nor '{\tt *}' in between).  The coefficients $+1$ and $-1$
are suppressed unless the exponent of the variable is $0$ in which
case the variable is suppressed. The caret '\verb@^@' is used to
indicate exponentiation.  Exponents with the value $1$ are suppressed
and if a variable has the exponent $0$ then the variable is
suppressed.  These rules apply recursively to the coefficients which
may themselves be polynomials.  A few examples are in order.

\begin{center}
\begin{tabular}{|c|c|} \hline
recursive polynomial & external canonical form \\ \hline
 & \\
$ -x^4 + 2x^3 - x + 3 $ & \verb@(-x^4+2x^3-x+3)@ \\
 & \\
$ (x^2+1)y^3+(x+8)y-5 $ & \verb@((x^2+1)y^3+(x+8)y+(-5))@ \\
 & \\
$  - (x^2-4)y^4 + y^2 - y - x$ & 
  \verb@((-x^2+4)y^4+(1)y^2+(-1)y+(-x))@ \\ 
 & \\ \hline
\end{tabular}
\end{center}

\noindent
Note that a constant polynomial in $r$ variables will be enclosed in
$r$ sets of parentheses.  For example, the constant polynomial $2$ in
3 variables will be represented in external canonical form as
\verb@(((2)))@.

The algorithm {\tt IPREAD} reads an $r$-variate recursive polynomial
over $\BbbZ$ in external canonical form from the input stream. The
polynomial that is read is stored in internal canonical form and the
number of variables is also recorded.  The variables are {\em not}
stored.  Integer coefficients may be of arbitrary length but exponents
must be {\tt BETA}-digits.  Since no sorting is performed on the
terms, they must be given in order of descending degree.  This is an
important remark since almost all algorithms that manipulate
polynomials require that the terms be ordered and violating this rule
will undoubtedly cause incorrect results to be computed and may even
crash the system.  Another important remark is that terms whose
coefficients are 0 should not be given as these terms will be stored
and may cause problems, for example in equality testing.

Although {\tt IPREAD} is happiest when a polynomial is given in
external canonical form as exemplified by the previous examples, the
user is allowed some freedom.  An arbitrary number of spaces may
interspersed between the coefficients, the variables, the exponents
and the symbols '\verb@+@', '\verb@-@' and '\verb@^@'.  Spaces may not
be inserted within a variable nor within an integer.  Coefficients
with magnitude 1 as well as the exponents 0 and 1 may be explicitly
given. Thus, for example, \verb@((x ^ 2+1) y^3+(1x+8) y^1-(5x^0) y^0)@
is perfectly valid and is equivalent to the second example given in
the table above.

Since {\tt IPREAD} was intended to be used mainly for reading output
produced by previous computations, it is designed to be fast and,
consequently, very little error checking is performed on the input.
Among other things, {\tt IPREAD} does not check for consistency among
the variables, e.g. \verb@((y)x^2+(z)y)@ will be accepted as valid
input and would be identical to
\verb@((u)v^2+(u)v)@ in internal representation.  Also, {\tt IPREAD}
does not check for consistency among terms, i.e.  each term is
processed separately and it is not checked whether all terms have the
same number of recursive nestings.  For example, \verb@(y^3+(x-1)y)@
will be accepted although the first term, \verb@y^3@, is a univariate
polynomial whereas the second,
\verb@(x-1)y@, is a bivariate polynomial.  {\em It is therefore the
responsibility of the user to see that polynomials are input properly.}

The algorithm {\tt IPWRITE} takes as inputs an $r$-variate recursive
polynomial \verb@A@ over $\BbbZ$ and a list
\verb@V = (@$v_1,\ldots,v_r$\verb@)@ of $r$ variables and writes \verb@A@ to
the output stream using the variables specified with $v_r$ as the main
variable and $v_1$ as the most minor variable.  The list \verb@V@ may be
initialized using {\tt VLREAD} which reads a variable list from the input
stream.  For generating a list with a fixed number of variables one could also
use an expression such as {\tt LIST3(LFS("X"),LFS("Y"),LFS("Z"))}.  Here the
functions {\tt LFS} is used for converting a C string to a SACLIB variable.
It is possible to use the algorithm {\tt IUPWRITE} to write univariate
recursive polynomials but this algorithm was intended mainly as a subroutine
to be called by {\tt IPWRITE}, which also handles univariate polynomials, and
the user need not even be aware of its existence.

There is an additional set of input functions of which the top level function
is {\tt IPEXPREAD}. The format accepted by this function is a bit more
convenient as expressions may be of the form 
\verb@(3 X Y^2 + X)^3 - (Y X + Y) (X - 1)^2 + 5@.
Note that {\tt IPEXPREAD} also takes a variable list as input and therefore
can detect the order of the variables without requiring the recursive
structure made explicit by parentheses.

To be more precise, {\tt IPEXPREAD} accepts any polynomial expression built
from integers and variables using +, -, blanks for multiplication, \^\ for
exponentiation, and parenthesis for grouping. The expression may be
terminated by any character not being part of the legal input set (e.g.\ a
period, a semicolon, etc.). This terminating character is not removed from
the input stream.

The function {\tt IPEXPREADR} has the same specification as {\tt IPEXPREAD},
with the difference that it {\em does} remove the terminating character.


\subsection{Recursive polynomials over $\BbbQ$}
For $r$-variate recursive polynomials over $\BbbQ$ the algorithms {\tt RPREAD}
and \linebreak {\tt RPWRITE} are the corresponding input and output routines.
The situation for rational polynomials is essentially the same as that for
integral polynomials with the exception that the base coefficients may be
rational numbers.  The same freedoms on valid input apply and an arbitrary
number of spaces may be inserted before and after '\verb@/@'.  If the
denominator of a base coefficient is 1 then only the numerator is in the
external canonical representation.  As an example, the external canonical
representation of $\frac{2}{7}x^3-65x^2+5x+\frac{12}{4}$ is
\verb@(2/7x^3-65x^2+5x+12/4)@ which, among many other possible
variations, may be input as \verb@(2/7x^ 3- 65 x^2 + 5/1x+12/ 4)@.  It
should be noted that the rational base coefficients are not reduced to
lowest terms when converted to internal representation.  Corresponding
to {\tt IUPWRITE} is {\tt RUPWRITE} which, again, need not concern the
user.

For rational polynomials there are also input functions for reading polynomial
expressions. Here the name of the top level function is {\tt RPEXPREAD}. The
input format here is the same as in the integral case, except that numbers may
be rational.

\subsection{Distributive polynomials over $\BbbZ$}
The external canonical representation of sparse {\em distributive}
polynomials over $\BbbZ$ is as follows.  The entire polynomial is
enclosed in parentheses.  Each term is made up of the integer
coefficient followed by the variables having positive exponents with
each variable separated from its corresponding exponent by a caret.
The coefficient and each variable-exponent pair is separated by a
single space.  As was the case for recursive polynomials, coefficients
and exponents with a magnitude of 1 are suppressed as are variables
with exponent 0.  For example, the polynomial $2x^3y^5-xy^3-4y+x+1$
will be expressed in external canonical form as
\verb@( 2 x^3 y^5 - x y^3 -4 y + x +1 )@.

The algorithms {\tt DIIPREAD} and {\tt DIIPWRITE} are the input and
output routines for distributive polynomials over $\BbbZ$.  {\tt
DIIPREAD} takes as input a variable list \verb@V@ =
\verb@(@$v_1,\ldots,v_r$\verb@)@ and reads a distributive polynomial
in external canonical form from the input stream.  The ordering of the
variables in \verb@V@ is significant and the variables in each term of
the polynomial that is read must appear in the same order that they
appear in \verb@V@ and the terms must be ordered in descending degree
in $v_r$.  For example, if\space \verb@V@ = \verb@(x,y,z)@ then \verb@(4 z^5
- y^2 z^4 + 9 x y z)@ is valid but \verb@(4 z^5 + 9 y x z - y^2 z^4)@
is not for two reasons---first, \verb@y@ appears before \verb@x@ in
the term \verb@9 y x z@ and second, the term \verb@9 y x z@ appears
before \verb@- y^2 z^4@ which violates the rule that terms must appear
in order of descending degree in \verb@z@.  Additionally, if there are
two terms with the same degree in $v_r$ then they should be ordered
according to descending degree in $v_{r-1}$ and so on.  Coefficients
may be separated from the variables by an arbitrary number of spaces
(including no space at all).  Variables must be separated by at least
one space if there is no exponent explicitly given, otherwise an
arbitrary number of spaces may separate them.  For example \verb@(4z^5
- y^2z^4 + 9x y z)@ is valid but \verb@(4z^5 - y^2z^4 + 9xyz)@ is not
since
\verb@xyz@ will be treated as a single variable.  


\subsection{Distributive polynomials over $\BbbQ$}
Distributive polynomials over $\BbbQ$ may be read in and written out using
the algorithms {\tt DIRPREAD} and {\tt DIRPWRITE}.  The only difference
between rational distributive polynomials and integral distributive
polynomials is that the base coefficients may be rational numbers and
not just integers.  It should be clear after reading the preceding
subsections what constitutes valid input and we will not discuss this
matter further.

\subsection{Conversion Between Recursive and Distributive\protect\newline
            Representation}
Converting recursive polynomials to distributive polynomials can be
achieved by using {\tt DIPFP} which, given a polynomial in recursive
internal representation, computes an equivalent one in distributive
internal representation.  In the other direction, namely to convert
from distributive to recursive representation, the algorithm {\tt
PFDIP} is provided. Both {\tt DIPFP} and {\tt PFDIP} work for
polynomials over either $\BbbZ$ or $\BbbQ$ but the number of variables
must be specified.  For example, if \verb@A@ is a polynomial over
$\BbbQ$ in internal recursive representation and the user wants to
display \verb@A@ in external distributive representation then the code
\begin{center}
\verb@DIRPWRITE(r,DIPFP(r,A),V);@ 
\end{center}
where \verb@r@ is equal to the number of variables and \verb@V@ is a
list of \verb@r@ variables, will suffice.  

\subsection{Polynomials over $\BbbZ_m$}
The input and output routines for polynomials over $\BbbZ$ work
equally well for polynomials over $\BbbZ_m$.
